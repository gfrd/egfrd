\documentclass[a4paper, 11pt]{article}
\usepackage{amsmath}
\usepackage{amsfonts}

\title{Realization of the main eGFRD algorithm in the current python code}
\author{Laurens Bossen}

\begin{document}
\maketitle

A number of independent processes or 'coordinates' take place inside a Domain. Independent processes are:
-Diffusion in a number of dimensions
-Decay reactions

When one of the processes in the domain has produced the current event, then a fire\_domain is executed. However, when an other
event external to the domain is the current event but the processes need to be propagated anyhow (a burst), then a burst\_domain
is executed.
Fire happens when the domain produced the top event of the scheduler. A burst event happens when the event associated with the
domain is still in the scheduler.
Fire->Event is no longer in scheduler
Burst->Event is is still in scheduler


Fire\_Single
-process event causing coordinate (a coordinate is an independent process in a domain, eg diffusion, decay)
-process all other coordinates

Burst\_Single
-process all coordinates (event was external)

After a fire or burst the domain is 'reduced' to a default NonInteractionSingle and a new domain (Pair, Interaction,
NonInteractionSingle) should be made.

Then reschedule the event


Note that from a Pair, Interaction (Multi) after a Fire or Burst you always 'go back' to a NonInteractionSingle. This means
that you change domain.
A speedup trick for this is to 'save' the original single in the Pair or Interaction domain so that we can potentially reuse
them when a fire or burst event takes place

In the old code, the processing of the coordinates is dealt with differently. When possible the particles in the domain are
first 'propagated'. This means that the particles in the domain are propagated to their new positions and that then new
zero shells are made for the particles. The putative decay reaction is then handled separately after the 'propagation'. In this
way the processes are modular. Similarly we could perform an interaction reaction after performing the propagation of the particle
towards the surface.
Doing a pair reaction however requires a different procedure since the two particles cannot be propagated on top of eachother.

Since the procedures of firing and bursting are similar, the current approach leads to a lot of code duplication. Since we also
intoduced a great number of new domains with different interactions, the number of combination grow introducing even more code
duplication.
To keep the code clear, we chose to generalize the procedure and perform the 'propagation' and 'reaction' steps in one step. In
this way also the pair reaction fits in the same framework.
This means that we first calculate the new position(s) of the particle(s) and directly perform the putative identity change. So,
we do not make new domains after calculating the new positions, but only do this after the putative reaction has taken place.

It is important to note that the reaction can fail because there is no space for the product particles. Currently this is caught
by an exception (the procedure for the decay reaction fails) and no changes are made to the system. In the new framework we will
have to handle it differently. We will need to make sure that the positional change still gets processed (the propagation) whereas
the identity change (the reaction) is not, the identity of the particle(s) stays the same.

A lot of the procedures have to be generalized anyway to facilitate the different dimensions. Also the fire\_pair method needs to
facilitate a mixed pair and therefore needs to be generalized to avoid code duplication.
Also the procedures are all very similar and should be easily generalized. This will make the code clearer and easier to understand.

Originally, the pair consists of two singles. These are also used after the propagation and are only removed when a reaction takes
place. There is in a sence a sort of a cache of the domains from which the pair domain was formed. When the event of the pair is then
processed we can make use of the cached singles to make the domains for the product particles. In the old structure, the singles
are also used to draw the decay times for single particles.
In the new framework we can reuse the singles if the particle identity has not changed, otherwise reuse is not possible. The domains
are coupled to the particles (their size, structure and reactionrules).



Processing the event means:
if Single && zero-shell && D != 0 then
  the single was a 'special' single (zero shell) -> make\_new\_domain
else
  the single was a 'normal' domain and event should be processed

  (calculate new position(s))
  if (D != 0) && dt>0 (time has passed)
    (there was actual relevant diffusion)
    determine new position(s) of particle(s)
    apply boundary conditions
    -check if still in shell
  else
    (no diffusion has taken place)
    positions are old position(s)


  (determine identitie(s))
  remove old domain (domain should be removed because otherwise it interferes with
  bursting during the 'reaction process'

  if event is Single\_Reaction, IV\_Reaction, IV\_Interaction
    get new identity for the resulting particle
    calculate new position for particles (for example Interaction, Decay reactions)
    apply boundary conditions to new position(s)
    actually move/remove/add the particle (update\_world)

  else
    (event was Burst or one of the possible Escape events)
    positions are 'old' positions
    particle identities are old identities
    particle number is old number

    apply boundary conditions to position(s)
    check that particles do not leave domain and do not overlap with other particles
    move particles to new positions (make change in the world)

  (we now have n particles in the world with certain identity and position)
  (In principle all particles should still be in the domain)

  for all particles
    if particle\_id is old\_particle\_id
      NonInteractionSingle = cached NonInteractionSingle
      re-initialize time and size of NonInteractionSingle(s)
      add to shell container and domain hash
      log reused NonInteractionSingle
    else
      create default NonInteractionSingle for particle
      log newly created NonInteractionSingle
    schedule domain



Note that when a Pair or Interaction is made, the shells and domain\_id are removed from the shell container and
domain hash respectively. This doesn't mean that the objects too have gone. The NonInteractionSingles are subsequently
cached in the made domain (Pair, Interaction, Multi?). When the made domain is now fired --the event produced by the
domain is processed-- and the resulting particle has not changed its identity, then the cached domain can be used. The
shell of the cached domain will recreated in the shell container and its domain\_id will be added to the domains hash.



Create a Protective Domain:
-generate identifiers for domain and shell
-make domain object (instantiate with particle, shell dimensions, structure/surface, reaction\_rules, misc info)
 -make appropriate shell (done by constructor or later in case of Multi)
-place shell in container (move\_shell)
-
\end{document}
