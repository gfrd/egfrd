\documentclass[a4paper,11pt]{article}
\usepackage{amsmath}
\usepackage{amsfonts}

\title{Algorithm for deciding on, and dimensioning a new shell}
\author{Laurens Bossen}


\begin{document}
\maketitle

\section{Introduction}
The eGFRD algorithm works with protective domains. In each domain one or two partcles can exist. Also a surface can be
involved and the particles can live on different structures or surfaces. The domains produde event that are organized
in the scheduler.

When an event from the scheduler is processes, the changed position (and potentially its identity) of the particles is
processed and a new domain for the particles has to be formed. The processing of the event
namely leaves the particles in the old domain as NonInteractionSingles with a size equal to the size of the particle
(they are zero-dt domains). The system will now decide what domain to make next.

There are a number of different domains that can be created from a NonInteractionSingle, and we need to decide which one is
the most appropriate. The options for a domain in descending order of computational cost per unit clock time are:
\begin{itemize}
 \item A NonInteractionSingle (a single particle in a symmetric domain modeling diffusion and potential decay of the
			       particle)
 \item An InteractionSingle (a single particle in the 3D space in a cylindrical domain modeling the diffusion of the particle
			     and non-specific binding of the particle to a surface).
 \item A Pair (two particles potentially on two different structures in a domain modeling diffusion and reaction between
	       the particles).
 \item A Multi (any number of particles in a collection of connected shells performing Brownian Dynamics)
\end{itemize}

\paragraph{Goal of algorithm}
The goal of the algorithm is to minimize the total time spent on computations while still allowing for reactions between
particles and non-specific association of a particle with a surface such as the membrane or DNA. This means in practice
that we want to make NonInteractionSingles for as long as possible and that only when a reaction or interaction becomes
probable switch to
making Pairs and InteractionSingles. We also want to defer the creation of a Multi a long a possible, since this really
takes up a lot of computation time.

Also we don't want to waste investments. This means that when we have just made a new
shell, we don't want it to be bursted right away. In short:
\begin{itemize}
 \item allow reactions and interactions
 \item spend CPU time well
 \item don't waste CPU time
\end{itemize}

There two different classes of domains, NonInteractionSingles and the rest. The rest is composed of
Pairs, Interactions and Multis. The separation in these two classes is caused by the fact that the NonInteractionSingle
is the default domain since they are the most efficient, and from the NonInteractionSingles we can go to other domains.
Also, the most appararent property of the NonInteractionSingle is then that it can 'burst' other domains to initiate the
creation of a more complex domain, typically a Pair or Interaction.
The making of a new Domain such as a Pair/Interaction/Multi is therefore initiated by a NonInteractionSingle, and it is
not possible to go from an Interaction domain directly to a Pair domain, we first need to 'go back' to the
NonInteractionSingle.

\section{Algorithm}
\subsection{Burst or reaction volume}
First, we define a $burst volume$ or $reaction volume$. The
$reaction volume$ effectively defines the threshold for when the algorithm should stop trying to make NonInteractionSingles and
should start trying to do Interactions or Pairs.
More specifically, it defines a domain in which the particle can move, and when other shells enter into this space it is a
signal that a reaction or interaction between the particle and the intruding object is probable and that a Pair or Interaction
should be attempted.

This also means that the algorithm will make NonInteractionSingles as long as no other object intrude into its
$reaction volume$. Making the $reaction volume$ small, therefore makes sure that we use the efficient NonInteractionSingles
for as long as possible and that the algorithm will only attempt a Pair or Interaction when the particles get 'close'.

For three dimensionally diffusing particles, the threshold is simply a $burst radius$, since the diffusoin is in three
dimensions, and the protective domain is spherical.
For particles that only diffuse in two or one dimension, there is a certain direction in which the reaction takes place.
For these particles we define the $reaction volume$ in the same shape as its appropriate protective domain, where the
$reaction radius$ is in the same dimension as we would size its domain.

Note that since overlap calculation between spheres and cylinders are annoying, we approximate the $reaction volume$ using
a sphere. The approximation introduces some inefficiency, since space that is not relevant for the particle is also covered
by the spherical $reaction volume$. However, when the size of the $reaction radius$ is small (typically 1.1 times the radius
of the particle), then the effect should be only small.
Note also that although the approximation could in principle allow two shell to overlap, since the corners of the cylinders
are not covered by the sphere, it still makes sure that no particles
overlap because the particle cannot leave the sphere. In this sence, the approximation does not introduce any errors.


\subsection{Bursting and finding potential partners for domains}
When objects intrude into the $reaction volume$, the algorithm will burst the objects and will start to try
to make a more complex domain than a NonInterationSingle. Bursting of domains should not be indiscriminatively though.
When an intruding domain has just been made (no time has passed since the making of the domain), then the domain should not
be bursted.
This prevents us from going back to the situation before the domain was made, eliminating the potential for a cycle. A newly
made Pair, Interaction or Mixed Pair domain is always sized such that it will not directly be bursted (see below),
however, a NonInteractionSingle could be made in the $reaction volume$ of another NonInteractionSingle. In this case the
NonInteractionSingle will not be bursted when no time has passed since its creation.

However, when sizing a NonInteractionSingle we will burst intruding domains in which time has passed. This means that if the
domain is just freshly made, we assume that it has been done with some wisdom (socially), and we shouldn't throw away the
investment.

Also note that object such as surfaces and Multi cannot be bursted.

We can only partner up with objects whose position is exactly known. Objects whose position is exactly known are:
\begin{itemize}
 \item surfaces
 \item particles in Multis
 \item just bursted (initialized) NonInteractionSingles ($dt=0$ shells)
 \item 'dead' NonInteractionSingles (note that $dt=\infty$ for these shells)
\end{itemize}
The position of a surface or particles in a Multi is always known, so we can use them directly as partners in domain making.
The position of particles in a non-zero shell ($dt \neq 0$) is not exactly known, and therefore we first have to burst the
domain if we want to use its contents as partners. A burst will propagate the particles in the domain and make a
NonInteractionSingle domain for each particle.
In the same way we can burst 'dead' NonInteractionSingles, although little would change. The particle will stay in the
same position, just the $dt$ for the domain will be reset to zero, and a new event time will be drawn ($dt=\infty$) if the
particle was not used in the new domain.


\subsection{Deciding what domain to try, defining the horizon}
To define a threshold to determine if a reaction or interaction with an other object is a good idea, we define the $horizon$.
To form a new domain that includes an other particle or surface than the current one, the current particle 'looks around' in
its $horizon$ after bursting the domains in its $reaction volume$. Since partners can
not only be found in the same structure, but also in other structures, we look around in 3D. This means that the horizon
defines a sphere with $horizon$ radius.

The $horizon$ is different for different partners, but in general
is the distance between two objects (two particles for example) that is the sum of their $reaction radii$. The
reasoning is
that if the two particles are so far away from each other that two NonInteractionSingles can be made with sizes larger than the
$reaction volume$, then a reaction is no longer probable between the objects and making one (or two) NonInteractionSingles
would be better. However, if the $reaction radii$ still overlap, a reaction is deamed probably and a Pair, Interaction or Multi
should be attempted.

When we would make the $horizon$ smaller, the algorithm would still make NonInteractionSingles when the $reaction volumes$
overlap. After an event of one of the domains, this would also lead to a burst of the other domain without attempting a more
complex domain. This is inefficient use the domain. Making the $horizon$ larger would result in larger Pair or Interaction
domains which have a lower probability of success.

The $horizon$ varies for different partners. For the interaction of a 3D diffusing particle with a membrane for
example, a different threshold is used than for a reaction between two particles in 3D.
For a Pair in the same structure, we define the $horizon$
as the sum of the $reaction radii$, although this is possibly slightly different for a MixedPair domain.
The $horizon$ for an Interaction on the other hand, is defined as the $reaction radius$ of the particle since the surface is
immobile and the domain making decision lies solely with the particle.

Note that when a Pair or Interaction is not possible, the algorithm will fall back to making NonInteractionSingles.
However, when also NonInteractionSingles become inefficient and no Pair or Interaction can be made, the algorithm will need to
decide to swith to a Multi. The threshold for a Multi can also be regarded as a horizon since a Multi is only efficient when
the two objects are close enough.
The $horizon$ for a Multi therefore denotes when NonInteractionSingles are no longer efficient and a Multi should be
made to propagate the particle. In general the Multi horizon is really small reflecting the efficiency of the Green's functions
over brute force Brownian Dynamics.


\subsection{Sizing up different domains}
Above we discussed how we try to make certain domains.
When we try to make a certain domain, such as an Interaction, we need to know if the domain will fit in the current configuration
of domains in the system.
The different domain such as a 2D Pair, an Interaction or 3D NonInteractionSingle have different shapes and requirements, and
can be sized in different directions. The currently possible domains and their shapes are:
\begin{itemize}
 \item 1D NonInteractionSingle -> cylinder, scale in 1D ($z$)
 \item 2D NonInteractionSingle -> cylinder, scale in 2D ($r$)
 \item 3D NonInteractionSingle -> sphere, scale in 3D ($r$)
 \item Pairs -> same shape as NonInteractionSingle, minimal size
 \item Interaction -> cylinder, scale in 3D ($r$ and $z$), minimal size, stick to membrane or rod
 \item Mixed Pair -> cylinder, scale in 3D ($r$ and $z$), minimal size, stick to membrane
 \item Multi -> same as NonInteractionSingle but fixed in scalable coordinate
\end{itemize}

\paragraph{Sizing up an Interaction, Pair or Mixed Pair}
Although the shapes and requirements for the domains are all quite different, there are some fixed rules for the maximum size
of the domains.
When we are making a Pair, Interaction or Mixed Pair, we want to make it such that NonInteractionSingles do not have the
tendency to burst this domain. The reasoning is that we are making the current domain because a reaction or interaction
between the object is probable and other particles should be kept at a distance. Sizing such a domain into the $reaction
volume$ of a NonInteractionSingle suggests that this NonInteractionSingle should attempt to make a reaction with the contents
of the current domain, which is against making the domain in the first place.

So, since we don't want to throw away the investment in the current shell, and we decided that it was a good idea to make this
domain in the first place, we leave all NonInteractionSingles at a distance such that the $reaction volume$ of the
NonInteractionSingle in question does not overlap with the new domain. This way these Single won't immediately try to
interfere and burst the newly made domain.
Note that domains on a surface are only sized in the coordinates 'in' the surface and that NonInteractionSingles in the bulk
are mostly irrelevant.

As an approximation, the $reaction volume$ of the NonInteractionSingle is chosen to be spherical also in the case of a 1D or
2D diffusing particle
(see above). This means that also when making a Pair, Interaction or Mixed Pair, the particles in a NonInteractionSingle will
be kept at a distance of $reaction radius$.

When the nearest relevant object is an object other than a NonInteractionSingle, such as a surface (or 'dead'
NonInteractionSingle), a
Pair, Interaction or Multi domain, then the maximum size of the domain is the distance to the object, since these objects do
not have a $reaction volume$ and cannot burst the domain directly.

\paragraph{Sizing up NonInteractionSingles}
To size up a NonInteractionSingle, the rules are similar.
The maximum size of a NonInteractionSingle is the distance to the nearest relevant object when the nearest relevant object is
a surface (or 'dead' NonInteractionSingle), Pair, Interaction or Multi.
In the case the nearest relevant neighbor is another NonInteractionSingle we should be more careful
and leave enough space for the other one (the maximum size is then for example half-way), especially if the $reaction volumes$
of the two NonInteractionSingles overlap.
This is to prevent unnecessary mutual bursting of the two NonInteractionSingles.

Note again that the nearest relevant object is the object
that is met first when scaling the domain in its coordinate. This is for example the $r$ coordinate for a 2D
NonInteractionSingle and the $z$ coordinate for a 1D NonInteractionSingle.
Note also that when the closest relevant Pair or Interaction domain was just made (no time has passed since its creation), it is
trivial for a NonInteractionSingle to make a domain of at least the $reaction volume$, since the $reaction volume$ was taken
into account when the Pair or Interaction was made (see above).

\paragraph{Sizing up Multis}
The shell for a particle participating in a Multi should be at least the size of the Multi horizon. The reasoning is that
when the particle leaves its shell it is at least horizon away from the multi. Making the shell smaller that the horizon,
will cause repeated re-creation of the Multi, since the particle can leave its domain but is not yet $horizon$ away from the
Multi. Making the shells bigger than the horizon will cause more Multi-ing than necessary to get out of the Multi horizon.
Making the shell exactly the size of the Multi horizon will then make sure that the Multi is not used longer than strictly
necessary.

Note that the shape of the shells for the particles in the Multi depends on the dimensions in which the particle can diffuse,
similarly as the domains for NonInteractionSingles and Pairs. This means that for a particle in
the bulk the shell is a sphere, for a particle in 2D and 1D it is a cylinder. As an approximation, however, we currently use
spheres as shells for all particles in a Multi.


\subsection{The algorithm in short}
From a NonInteractionSingle that has just been processed we will make a new domain. The NonInteractionSingle is the default
domain from which other domains can be formed. To decide what domain to make we use the following algorithm.
\begin{enumerate}
 \item First we check if there are any intruders in our $reaction volume$. The $reaction volume$ has in principle the same
       shape as the shell of the NonInteractionSingle, but we approximate it with a sphere. Note that these can be Singles,
       Pairs, Multis and even surfaces. If there are intruders, then this means that an other particle or surface so close
       in the relevant direction that a reaction or
       interaction is probable and that we should start thinking about making Interactions, Pairs or Multis.
 \item burst the 'burstable' intruders within the $burstradius=reaction volume$ with the exception of domains in which no
       time has passed yet (not burstable are: Multi, Surface, zero-shell NonInteractionSingles ($dt=0$ or 'dead')).
 \item compile a list of neighboring potential partners in all three dimensions (Multis, surface, zero-shell
       NonInteractionSingles).
 \item get the closest potential partner and decide on the domain to make. Note that the closest potential partner does not
       always leads to an allowed
       interaction. A combination of a 1D diffusing particle and a 3D diffusing particle is for example not possible.
 \begin{itemize}
  \item if the closest potential partner is a surface AND is within the $surface horizon$, then try an Interaction.
  \item elseif the closest potential partner is a NonInteractionSingle AND within the Pair horizon, then try a Pair
  \item if the above fails, then the closest partner is a surface AND outside the surface horizon OR a NonInteractionSingle
        AND outside the Pair horizon OR the domain could not be made.
  \item If the closest potential partner is further than the Multi horizon, then make a NonInteractionSingle.
  \item else make/join a Multi.
 \end{itemize}
 \item Make sure that the domain has the proper size.
\end{enumerate}



\section{Notes on special domains}
Note that a surface behaves similarly to a stationary (D=0), non-decaying particle. It doens't produce any events and is
therefore never fired and never bursts other domains. Since it is stationary it also doesn't contribute to the horizon.

A zero shell is made in a number of situations. They are made when a new particle is introduced in the simulation system
(for example at the beginning of a simulation run), or when a shell is bursted and is reduced to a collection of zero-dt
NonInteractionSingles.
Since zero shells have a dt of zero their events are all at the top of the
scheduler, and consequently these events will be processed before and other events. Note that these shells always produce
an $ESCAPE$ event, and that there can be only other zero shells in the simulation system if the shell that is associated with
the currently event is also a zero shell.


\end{document}
